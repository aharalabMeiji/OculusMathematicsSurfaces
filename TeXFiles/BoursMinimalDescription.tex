\documentclass[12pt,dvipdfmx]{article}
\usepackage{mathrsfs}
\usepackage{graphicx}
\usepackage{amsmath}


\begin{document}

\begin{center}

{\bf \Large Bour's minimal surface}

\end{center}


\section{Abstract}

 Bour's minimal surface is a two-dimensional minimal surface, embedded with self-crossings into three-dimensional Euclidean space. It is named after Edmond Bour, whose work on minimal surfaces won him the 1861 mathematics prize of the French Academy of Sciences. (from Wikipedia)

The surface can also be expressed as the solution to a polynomial equation of order 16 in the Cartesian coordinates of the three-dimensional space. (from Wikipedia)



\section{Definition}
The points on the surface may be parameterized in polar coordinates by a pair of numbers $(r,\theta)$. Each such pair corresponds to a point in three dimensions according to the parametric equations
\begin{align*}
x&=	r \cos \theta-\frac{1}{2} r^2 \cos(2\theta)	
\\
y &=	-r \sin\theta-\frac{1}{2} r^2 \sin(2\theta),	
\\
z&=	\frac43 r^{\frac{3}{2}} \cos(\frac32 \theta) 	
\end{align*}

for $\theta$ in $[0,2\pi)$ and $r$ in $[0, \infty)$.


\begin{thebibliography}{9}

\bibitem{Wikipedia} Wikipedia \verb|https://en.wikipedia.org/wiki/Bour's_minimal_surface|

\bibitem{MathWolrd} \verb|http://mathworld.wolfram.com/BoursMinimalSurface.html|


\end{thebibliography}


\end{document}
