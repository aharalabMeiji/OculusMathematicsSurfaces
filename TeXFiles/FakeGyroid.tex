\documentclass[12pt,dvipdfmx]{article}
\usepackage{mathrsfs}
\usepackage{graphicx}
\usepackage{amsmath}


\begin{document}

\begin{center}

{\bf \Large Fake GYROID}

\end{center}


\section{Abstract}

The gyroid is an infinitely connected periodic minimal surface containing no straight lines (Osserman 1986) that was discovered by Schoen\cite{Schoen}. Gro\ss e-Brauckmann and Wohlgemuth (1996) proved that the gyroid is embedded. (from MathWolrd)

The gyroid is the only known embedded triply periodic minimal surface with triple junctions. In addition, unlike the five triply periodic minimal surfaces studied by Anderson et al. (1990), the gyroid does not have any reflectional symmetries (Gr\ss e-Brauckmann 1997). (from MathWolrd)

\section{Definition}

The fake version is based on the equation 
\[
\cos x \sin y + \cos y \sin z + \cos z \sin x = 0
\]
 that gives a non-minimal surface close to the true gyroid. (from MathCurve)




\begin{thebibliography}{9}

\bibitem{Mathworld} MathWorld bt Wolfram, \verb|http://mathworld.wolfram.com/Gyroid.html|

\bibitem{Schoen} Schoen, A. H. "Infinite Periodic Minimal Surfaces Without Selfintersections." NASA Tech. Note No. D-5541. Washington, DC, 1970.

\end{thebibliography}


\end{document}
