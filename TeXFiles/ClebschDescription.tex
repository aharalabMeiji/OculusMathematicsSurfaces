\documentclass[12pt,dvipdfmx]{article}
\usepackage{mathrsfs}
\usepackage{graphicx}
\usepackage{amsmath}


\begin{document}

\begin{center}

{\bf \Large CLEBSCH surface}

\end{center}


\section{Abstract}

The Clebsch surface, (or Clebsch diagonal cubic surface or Klein's icosahedral cubic surface) is a non-singular cubic surface, studied by Clebsch (1871)\cite{Clebsch} and Klein (1873), all of whose 27 exceptional lines can be defined over the real numbers. The term Klein's icosahedral surface can refer to either this surface or its blowup at the 10 Eckardt points. (wikipedia)

\section{Definition}

The Clebsch surface is the set of points $(x_0:x_1:x_2:x_3:x_4) \in P^4$ satisfying the equations
\begin{align*}
{ x_{0}+x_{1}+x_{2}+x_{3}+x_{4}=0,}\\
{ x_{0}^{3}+x_{1}^{3}+x_{2}^{3}+x_{3}^{3}+x_{4}^{3}=0.}
\end{align*}

This sculpture is a surface defined by an equation\cite{MathCurve}
\[
64  x^3 + 48  x^2  y- 192  z^2  x + 48  z^2  y
            - 31  y^3 - 54  y^2 - 24  y = 0.
\]

\begin{thebibliography}{9}

\bibitem{Clebsch} Clebsch, A. (1871), "Ueber die Anwendung der quadratischen Substitution auf die Gleichungen 5ten Grades und die geometrische Theorie des ebenen Fünfseits", Mathematische Annalen, 4 (2): 284–345, doi:10.1007/BF01442599

\bibitem{MathCurve} \verb|https://www.mathcurve.com/surfaces.gb/clebsch/clebsch.shtml|

\end{thebibliography}


\end{document}
