\documentclass[12pt,dvipdfmx]{article}
\usepackage{mathrsfs}
\usepackage{graphicx}
\usepackage{amsmath}


\begin{document}

\begin{center}

{\bf \Large Dupin Cyclide}

\end{center}


\section{Abstract}
Dupin cyclide or cyclide of Dupin is any geometric inversion of a standard torus, cylinder or double cone.  They were discovered by (and named after) Charles Dupin in his 1803 dissertation under Gaspard Monge. The key property of a Dupin cyclide is that it is a channel surface (envelope of a one-parameter family of spheres) in two different ways. This property means that Dupin cyclides are natural objects in Lie sphere geometry.

Dupin cyclides are often simply known as cyclides, but the latter term is also used to refer to a more general class of quartic surfaces which are important in the theory of separation of variables for the Laplace equation in three dimensions.

Dupin cyclides were investigated not only by Dupin, but also by A. Cayley, J.C. Maxwell and Mabel M. Young.

Today, Dupin cylides are used in computer-aided design (CAD), because cyclide patches have rational representations and are suitable for blending canal surfaces (cylinder, cones, tori, and others).  (Wikipedia)

\section{Definition}

The Cartesian parametrization of the cyclide by the double family of curvature circles is:

\[
\begin{aligned}
x &= \dfrac{d(c - a \cos u \cos v) + b^2 \cos u}{a - c \cos u \cos v} \\
y &= \dfrac{b\sin u (a - d \cos v)}{a - c \cos u \cos v} \\
z &= \dfrac{b \sin v ( c \cos u - d )}{a - c \cos u \cos v}
\end{aligned}
\]

(Mathcurve)



\begin{thebibliography}{9}

\bibitem{Mathcurve} 
\verb|https://www.mathcurve.com/surfaces.gb/cycliddedupin/cyclidededupin.shtml|


\end{thebibliography}


\end{document}